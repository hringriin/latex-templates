\documentclass{scrartcl}

\usepackage{qrcode}

% QR-CODE HEIGHT
%
\newcommand{\qrHeight}{0.5\textwidth}

% QR-CODE LEVEL
%
% The QR code specification (ISO 18004:2006) includes four levels of encoding:
% Low, Medium, Quality, and High, in increasing order of error-correction
% capabaility.  In general, for a given text a higher error-correction level
% requires more bits of information in the QR code. The key level=hlevel
% specificationi selects the minimum acceptable level. The hlevel specificationi
% may be L, M, Q, or H; the default is M. It may happen that the smallest QR code
% able to encode the specified text at the desired level is in fact large enough
% to provide a higher level of error-correction.  If so, qrcode automatically
% upgrades to the higher error-correction level, and a message is printed in the
% log file.
\newcommand{\qrLevel}{H}


% ENCRYPTION TYPE
%
\newcommand{\qrT}{WPA}


% SSID
%
\newcommand{\qrS}{MySSID}

% WIFI-PASSWORD
%
% The following characters needs to be escaped with a backslash (\)
%       #$&^_~% \{}
% This sequence includes a whitespace, which also needs to be excaped
%
% The `~` needs some special attention, it needs no backslash but the command
% `\string`: `\string~`
%
% For example the password
%   Th15~15#My&P455w0rd
% has to be written like this:
%   Th15\string~15\#My\&P455w0rd
\newcommand{\qrP}{MyPassword}


% SSID HIDDEN STATUS
%
\newcommand{\qrH}{false}

% OTHER QRCODE OPTIONS
%
\newcommand{\qrOptions}{}

\begin{document}
    \begin{center}
        { \Huge \textbf{WiFi Credentials} }
    \end{center}
    \vspace{1cm}
    \begin{table}[h]
        \begin{center}
            \begin{tabular}{rl}
                \multicolumn{2}{c}{
                    {
                        \qrset{height=\qrHeight,level=\qrLevel}
                        \qrcode[\qrOptions]{WIFI:T:\qrT;S:\qrS;P:\qrP;H:\qrH;}
                    }
                } \\\\ \hline \\
                \textbf{SSID:}          &   \qrS    \\
                \textbf{Password:}      &   \qrP    \\
                \textbf{Security:}      &   \qrT    \\
                \textbf{Hidden:}        &   \qrH    \\
            \end{tabular}
        \end{center}
    \end{table}
\end{document}
